% ************************** Thesis Abstract *****************************
% Use `abstract' as an option in the document class to print only the titlepage and the abstract.
\begin{abstract}

Bacteria can exploit gene expression noise to create 
phenotypic heterogeneity in a population, 
which may allow a fraction of the population to survive 
a sudden change in environmental conditions.
%%
The phenotypic variations are typically caused by stochastically 
turning on or off a specific set of genes.
%%
Alternative sigma factors 
(common regulatory proteins of bacterial stress responses) 
often drive this variability.
%%
Recent studies show that sigma factors may adopt dynamical behaviours, 
including stochastic pulsing and bistability, 
to modulate their cellular abundance.
%%
It has also been shown that biochemical ultrasensitivity 
is important for several behaviours, such as bistability, 
but its exact role in the sigma factor circuit remains unclear.
%%
Here I simulate a simplified mechanistic model of 
an alternative sigma factor circuit using the Gillespie algorithm.
%%
The model features a mixed self-activation and 
negative feedback loop with time delay.
%%
I first show that a range of dynamical behaviours is produced.
%%
Then I observe that without ultrasensitivity, 
several dynamical behaviours are significantly 
weakened or cannot be maintained.
%%
As many sigma factor circuits do not encode ultrasensitivity, 
it raises the question of how stochastic switching of 
gene expression is achieved.
%%
Sigma factors must bind to the RNA polymerase (RNAP) core enzymes to function, 
and recent studies show that the competition between sigma factors 
for the limited pool of  RNAP cores shapes sigma factor dynamics.
%%
In light of that, I propose two new mechanisms 
for a non-ultrasensitive sigma factor to maintain bistability.
%%
First, under strong competition, 
a bistable ultrasensitive circuit may force a non-ultrasensitive circuit 
to adopt bistability through the limitation of shared resources.
%%
Second, for two non-ultrasensitive circuits with low binding affinity 
of sigma factors to RNAP cores, the circuit is locked to 
a zero-state and turns on by rare binding events through a self-activation.
%%
I show that in the latter scheme, 
the variations in initial activation times are reduced 
as sigma factor-core RNAP binding affinity increases, 
which has implications for a previous model of heterogeneous activation 
of the alternative sigma factor $\sigma^V$ in 
\textit{Bacillus subtilis}.
%%
Since the core sigma factor circuit structure is conserved 
across many bacteria, this research may shed light on 
a general strategy for the bacteria population to create heterogeneity.
    
\end{abstract}
