%!TEX root = ../thesis.tex
%*******************************************************************************
%*********************************** Fourth Chapter *****************************
%*******************************************************************************

\chapter{Discussion}  %Title of the First Chapter

% \ifpdf
%     \graphicspath{{Chapter1/Figs/Raster/}{Chapter1/Figs/PDF/}{Chapter1/Figs/}}
% \else
%     \graphicspath{{Chapter1/Figs/Vector/}{Chapter1/Figs/}}
% \fi


%********************************** %First Section  **************************************
\section{Considerations on the model}

%% a bit context
The change of environment can be drastic and frequent considering
small size of bacteria,
which often demands the use of alternative sigma factors to adapt
to the new environment through a shift in the transcription profile.
%%
It is established that bacteria leverage noise to dynamically modulate
the expression of sigma factors and, thus, create phenotypic
heterogeneity inside the population to gain a survival advantage
\cite{locke11,cabeen17,park18a,schwall21a}.
%%
Theoretical studies show that a common network motif combining 
a positive autoregulation and a negative feedback loop with delay
underpins these dynamics (\cite{schwall21a} and Torkel Loman's
unpublished work).
%%
% mechanistic, implies bottom-up (though the parameters are
% biochemically explicable, I'm not confidence in their accuracy)
In this study, I used a simplified, mechanistic model of the same 
general structure to reproduce diverse dynamical behaviours of
the sigma factors.
%%
I used the discrete Gillespie algorithm to correctly predict the 
low-molecule dynamics and properly reflect the role of noise
in the circuit.
%%
I also proposed two new mechanisms to address the bistability
suggested in non-cooperativity circuits \cite{schwall21a}
through competition, namely
%%
(\textit{a}) A competing cooperative sigma factor circuit forces
bistability via shared pool of RNAP.
%%
(\textit{b}) Bistability in a non-cooperative circuit
maintained by rare sigma factor-RNAP core
binding event due to low binding affinity.
%%
The theoretical work here could motivate future studies to alter the 
sigma factor-RNAP core binding affinity (e.g. via sigma factor mutants)
to discover potential new dynamical behaviours.
%% 
It may also provide new insights for synthetic biologists to engineer
new oscillators or bistable switches by rewiring the 
bacteria sigma factor circuits.
%%
Nonetheless, the model is built under considerable uncertainty of the
actual biochemical processes and trade-off between generality
and the amount of molecular details encoded.
%%
% this is a strange sentence. Behind is some degree of diffidence,
% nevertheless, it is difficult to gather the last pieces together
Here, I will elaborate some important decisions and assumptions
for the model, and several future directions.

\subsection{Arbitrary units of time}
\label{sec:arbitrary_units_time}

Although the model uses the absolute, discrete number of molecules
per cell as the concentration for all chemical species
to accurately account for stochastic fluctuations,
the units of time are arbitrary.
%%
Notice that in the model (Section~\ref{sec:low_CN}),
the maximum steady-state concentration and the molecular lifetime
is tuned independently through $\beta$ and $\tau$.
%%
When the circuit is activated, the expression rate factor $v_H$ is 
high and can approximate to 1.
Thus, the steady-state abundance of $\sigma$ is roughly $\beta$.
%%
% establish that \tau is the average lifetime
The degradation rate $1/\tau$ gives the molecule an average lifetime
of $\tau$. To support the argument, consider the (deterministic) first-order 
degradation reaction of molecule $x$ with degradation rate constant
$1/\tau$:

\begin{align}
    \frac{dx}{dt} = -\frac{1}{\tau}\cdot x
\end{align}

Given the initial concentration $x_0$ at $t = 0$, the evolution of $x$
along time is expressed as

\begin{align}
    x = x_0 \cdot e^{-t/\tau}
\end{align}

When $x_0$ is large enough, the average molecular lifetime ($\bar{t}$) can be 
approximated by the area under the $t$-$x$ curve divided by the
total number of molecules, $x_0$, namely

\begin{align}
    \bar{t} =& \frac{1}{x_0}\int_{t = 0}^{\infty}x_0 e^{-t/\tau}dt\\
        =& \tau
\end{align}

Which shows that in first-order degradation kinetics, the reciprocal
of the degradation rate is the average molecular lifetime.

In the model, time is measured relative to the average lifetime of
different molecules.
In all simulations, the lifetime of the sigma factor is 10 time units.
To reflect the delay from the anti-sigma factor, the lifetime of
$A$ is set to 50, 5-times of that of the sigma factor.
%%
% validate k_off
In the case of competition against the RNAP cores,
the dissociation rate constant $k_{off}$ is reported to be 
\SI{1e-3}{\per\second}, equivalent to the average lifetime
of $E\sigma$ being \SI{1000}{\second} \cite{mauri14}.
%%
The average lifetime of sigma factors is required to normalize
$k_{off}$ to the time units I use.
To begin with, the half-life of the major pool of proteome in $E. coli$ 
is reported to be around 1 hour.
From first-order degradation kinetics, we can derive that half-life
is $\ln 2 \cdot \tau$ (proof not shown) and thus,
the average lifetime is around \SI{5000}{\second} \cite{maurizi92}
%%
However, another study reports the half-life of bacteria sigma factor
under protease negative regulation to be 1 minute, 60-times faster than
the average proteome \cite{el-samad05}.
%%
Given the scarcity of measurement, I approximately set the average lifetime
of the holoenzyme to be the same as sigma factor (\SI{1000}{\second}).
%%
% finally, how to improve
The predictability of the model can be improved by dedicated measurements
of the half-lives of the holoenzyme and the sigma factor.
%%
Also, altering the relative lifetime of the anti-sigma factor compared
to the sigma factor could reveal interesting dynamics,
which contributes to future research directions.
%%
However, I argue that the accuracy of the simulation is not harmed
by the use of arbitrary time units, since I inspected that
the change of absolute lifetime of molecules does not change the 
steady-state distribution of sigma factors, given a small
time step used (data not shown).

% \subsection{Steps of reaction under steady-state assumptions}
\subsection{Derivation of model parameters}
\label{sec:discussion_parameters}

In this section, I will discuss the rationale of choosing other
model parameters besides the aforementioned degradation/dissociation
rates.
%%
% the dissociation constant is important
The dissociation constant between RNAP cores and sigma factors,
$K_{E\sigma}$, quantifies the binding affinity and plays an important
role in tuning the strength of sigma factor competition and,
thus, the existence of bistability in the system.
%%
The dissociation constant has the units of concentration
\cite{alon06}.
Since I used molecules per cell as the units for concentration
as many literature report dissociation constants in 
molar concentrations,
I will first determine the conversion between the two.
%%
% conversion between molar and molecules
% warning: the style of siunitx is a bit weird
Nanomolar (\si{\nano\Molar}) is the common units for the 
physiological concentrations of macromolecules.
Multiplying the unit concentration (\SI{1}{\nano\Molar}) by the average cellular
volume of bacteria (\SI{1.32}{\femto\liter}) \cite{mauri14} and the
Avogadro constant gives approximately 0.8 (molecules/cell).
%%
Maeda \textit{et al.} reported that the dissociation constants
of the alternative sigma factors in \textit{E. coli} range from
\SI{0.3}{\nano\Molar} to \SI{4.26}{\nano\Molar}, which translates to
0.24 to 3.41 molecules/cell \cite{maeda00}.
%%
However, a more recent study measures the binding affinity
by a different method in various cellular conditions.
The study displays that the dissociation constant
between one of the sigma factors, RpoH (alias $\sigma^H$ or $\sigma^{32}$) 
and RNAP cores
may range from 18 to 100 molecules/cell across different temperatures 
\cite{ganguly12}.
%%
Though both studies consistently report the binding affinity to
the housekeeping sigma factor to be higher than any of the alternative
ones, the absolute values can vary 10- to 100-fold, depending on
different methods and measuring conditions.
%%
I chose $K_{E\sigma}$ between 10 to 100 molecules/cell for most simulations,
in comply with the newer measurements.
%%
Notice that both studies measure the dissociation constants 
\textit{in vitro}. It would be interesting to see \textit{in vivo}
measurements, which may solve the conflicts and improve model accuracy.
%%
Also, since the $E$-$\sigma$ dissociation constant is shown to
change according to ionic strength and temperature.
Future study may explore whether the change in binding affinities
contribute to sigma factor competition.

% Finally, we talk about the amount of RNAP cores and 
% housekeeping sigma factors
The concentration of RNAP core enzyme in bacteria is also
an unsettled case, with estimates ranging from
approximately 1,500 to 13,000 copies per cell \cite{jishage95,grigorova06}.
%%
To ask the proportion of RNAP cores associated with
the alternative sigma factors, which is the number used in the model
to reflect housekeeping sigma factor competition,
I designed an equilibrium binding model.
%%
The total number of RNAP enzyme cores is set to $E_t = 2,000$.
Since the amount of housekeeping sigma factors (denoted by $\sigma^A$)
is reported to significantly exceed at approximately 3-fold,
I set $\sigma^A_t = 6,000$.
Finally, I chose the number of all alternative sigma factors to be
$\sigma^B_t = 2,000$ to reflect the observation that the alternative
sigma factors are fewer than the housekeeping ones.
%%
From the above discussion of binding affinities, I set
$K_{E\sigma^A} = 1$ and $K_{E\sigma^B} = 10$ to show the weaker 
binding affinities of the alternative sigma factors.
%%
The model is given by:

\begin{gather}
    K_{E\sigma^A} = \frac{E \cdot \sigma^A}{E\sigma^A}\\
    K_{E\sigma^B} = \frac{E \cdot \sigma^B}{E\sigma^B}\\
    E_t = E + E\sigma^A + E\sigma^B\\
    \sigma^A_t = \sigma^A + E\sigma^A\\
    \sigma^B_t = \sigma^B + E\sigma^B
\end{gather}

Where $E$ represents only the free RNAP cores, while $E\sigma^A$
are the ones bound to the housekeeping sigma factors,
and $E_t$ is the total amount.
The sigma factors adopt the same rules of notations.
%%
Solving the equations gives almost saturated RNAP core binging,
with $E\sigma^A \approx 1,910$ and $E\sigma^B \approx 89$
(molecules/cell).
%%
In the core model for the sigma factor circuit,
I used a smaller $E = 50$ (RNAP cores available to alternative 
sigma factors) in most settings
to simulate a fierce competition scenario.
%%
Altering the amount RNAP cores available to the alternative sigma
factors is shown to weaken the competition 
(Section~\ref{fig:forced_bistability}).
Future study may further quantify the effects and explore
whether bistability can sustain with changing total number of RNAP cores.

\section{Future work}

By simulating a simplified two-component mathematical model
of the alternative sigma factor circuit, I produced a range of
behaviours, some of which have been experimentally observed.
%%
From this model, I examined the scenario where two alternative sigma
factor circuits compete for the limited amount of RNAP cores.
From there I discovered two new bistable patterns of 
the sigma factor circuit.
%%
First, when the binding between sigma factors and RNAP cores
is strong, and if a ultrasensitive circuit exists in the cell,
another non-ultrasensitive circuit can be induced to be bistable.
The bistability could be transferred through the shared, limited
pool of RNAP cores.
%%
Second, when the binding affinity is weak (and if the amount of 
RNAP cores are limited due other sigma factors, especially
the housekeeping one), the event of holoenzyme formation may be
rare. Then, the system would be turned on by the rare binding event
and be turned off due to random walk to zero from a relative small number
of holoenzymes.

Considering that the research is theoretical, it would
be strengthened by evidence from experiments.
Furthermore, several experimental observations may provide the
model with new insights and in turn refine the theory.
%%
To begin with, there are still a few interesting directions that
one may explore under the current modelling framework.
%%
First, since the Gillespie algorithm generates correct low-copy number dynamics, 
it would be interesting to ask what is the effect of absolute 
number of molecules in the system to different dynamical behaviours.
%%
Experiments have shown that when noise is reduced due to an
increase in the absolute number of molecules, some dynamical 
behaviours, e.g., stochastic switching and stochastic pulsing,
can be weakened \cite{suel06, locke11}.
%%
It would be important to validate and quantify this effect in theory.
%%
Second, I observed that, similar to the effects of ultrasensitivity,
tuning the molecular lifetime of the anti-sigma factor to be
less than that of the sigma factor eliminates some behaviours,
including oscillation (data not shown).
%%
Intuitively, since the lifetime of anti-sigma factors determines the
time delay of the negative feedback in the model and the lifetime
of sigma factors determine the speed of positive autoregulation,
establishing the oscillation may require a relative slower
negative feedback.
%%
It will be interesting to confirm this from simulation or theory.

Several experiments may strengthen the foundation of the model,
including measuring the average lifetime of different sigma factors
(considering both protein degradation and dilution effects)
and the dissociation rate of the holoenzyme.
%%
A more important aspect is to measure the sigma factor-RNAP core
binding affinity (in the form of dissociation constant 
$K_{E\sigma}$) \textit{in vivo},
since this binding affinity has major implications to sigma factor
behaviours under competition.
%%
Moreover, there is still huge uncertainty in its measurement,
which is complicated by the dependence of $K_{E\sigma}$ on
different cellular conditions (Section~\ref{sec:discussion_parameters})
\cite{ganguly12}.
%%
Future research may pivot on whether stress influences the binding
affinity and, thus, alters sigma factor dynamics.
%%
The behaviour mapping in Figure~\ref{fig:different_behaviours}
predicts several dynamical behaviours of sigma factors that have
not yet been experimentally observed,
e.g. oscillation, stochastic anti-pulsing, etc.
%%
It would be possible to observe these behaviours using single-cell
techniques, by either constructing sigma factor or anti-sigma factor
mutants, or simply by tuning the strength of stress.
%%
The model also suggests that the strength of competition can modify
the bistable dynamics of non-ultrasensitive sigma factors.
%%
Relevant experiments can not only serve as a confirmation of the proposed
molecular mechanisms, but also help to discover new ways to tune
the phenotypic variability of a bacterial population.
%%
For example, the strategy which \textit{B. subtilis} tunes the variability
of $\sigma^V$ by the strength of stress is qualitatively reproduced by my model
\cite{schwall21a}.
%%
Finally, the range of dynamical behaviours shown in sigma factors may 
contribute to the synthetic biology toolbox.
%%
Orthogonal sigma factor circuits from other species
can be leveraged as a frequency-tunable stochastic pulsing generator
or as a source of heterogeneity in a population.

% \subsection{Heterogeneous activation time of sigV}
% \subsection{Transient single pulsing of sigB}
