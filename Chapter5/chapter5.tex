%!TEX root = ../thesis.tex
%*******************************************************************************
%*********************************** Fifth Chapter *****************************
%*******************************************************************************

\chapter{Conclusion}  %Title of the First Chapter

% \ifpdf
%     \graphicspath{{Chapter1/Figs/Raster/}{Chapter1/Figs/PDF/}{Chapter1/Figs/}}
% \else
%     \graphicspath{{Chapter1/Figs/Vector/}{Chapter1/Figs/}}
% \fi


%****************all in one section*********************

Noise unavoidably arises from the thermal motion of molecules and
from the fluctuating environment where the cells reside.
However, more and more evidence has shown that noise may play functional
role in biology \cite{raj08,eldar10a}.
%%
Bacteria can utilize noise to activate a specific set of genes
in a fraction of the population to create phenotypic 
heterogeneity.
%%
In this way, bacteria may "hedge their bets" to make some individuals
survive a sudden change in the environment or benefit the individuals
that do not switch on stress response (thus less metabolic burden)
if the favourable environment recovers
(Section~\ref{sec:functional_role_noise}).
%%
In bacteria, the activation of a certain group of genes is typically directed
by the expression of alternative sigma factors.
%%
Previous studies have shown that sigma factors may exhibit dynamical
behaviours upon stress, e.g. stochastic pulsing ($\sigma^B$ of
\textit{B. subtilis}) and heterogeneous start of activation
($\sigma^V$ of \textit{B. subtilis}) \cite{locke11,schwall21a}.
%%
In this study, I simulated a simplified mechanistic model 
with the Gillespie algorithm and produced
a wide range of behaviours, e.g., stochastic pulsing, stochastic
state-switching, stochastic anti-pulsing (i.e. inactivation pulsing),
oscillation, etc (Section~\ref{sec:general_diff_behaviours}).
%%
The model features mixed positive autoregulation and negative 
feedback loop (\cite{schwall21a} and Torkel Loman's unpublished work)
(Section~\ref{sec:low_CN}).
%%
Biochemical ultrasensitivity has been shown to facilitate some
dynamical behaviours, including oscillation and stochastic state-switching
\cite{ferrell14c}.
%%
However, the role of ultrasensitivity in the alternative sigma factor 
circuit remains unclear.
%%
By simulating a non-ultrasensitive circuit ($n = 1$), I observed that
not only is oscillation lost, but also other behaviours, 
e.g. stochastic switching and stochastic anti-pulsing, are significantly
weakened by increased fluctuations
(Section~\ref{sec:us_for_bs_and_oscillation}).
%%
Though ultrasensitivity can be important for bistability,
many sigma factor circuits show no known source of ultrasensitivity
from binding cooperativity.
%%
This suggests that there can be other sources of ultrasensitivity.
%%
It has also been shown that sigma factors are in competition with each
other for the limited amount of RNAP cores 
(Section~\ref{sec:intro_sigma_competition}).
%%
In light of that, I simulated a dual sigma factor model to reflect
the competition between them (Section~\ref{sec:sigma_competition_model}).
%%
Based on the model, I proposed two new mechanisms for bistability to
emerge from a non-ultrasensitivity circuit.
%%
First, when competition is strong, one ultrasensitive, bistable sigma factor
circuit can force another non-ultrasensitive circuit to show
bistability through the limitation of a shared pool of RNAP cores
(Section~\ref{sec:forced_bistability}).
%%
Second, when the binding affinity between the non-ultrasensitive
sigma factor and the RNAP core is low,
the abundance of sigma factors is trapped in a strict-zero state.
The circuit is then turned on by rare binding events and turned off
by the random walk from a low copy number to zero
(Section~\ref{sec:both_non_coop_bistability}).
%%
In the second scheme, I also showed that the activation variability caused by
bistability increases as the binding affinity decreases,
which qualitatively reflects the observations of $\sigma^V$ in 
\textit{B. subtilis} \cite{schwall21a}.
%%
As the core structure used in the model is shared among various bacterial
alternative sigma factor circuits,
these mechanisms may represent a general strategy to generate 
heterogeneity in a population.
%%
In future studies, it would be interesting to experimentally explore other
dynamical behaviours predicted by the model.
%%
It would also be important to validate the proposed mechanisms of 
non-ultrasensitive bistability by experiments.
%%
Finally, considering the wide range of dynamics produced from the
rather simple sigma-anti-sigma circuit, engineering orthogonal sigma factor
circuits may enrich the toolbox for synthetic biology.
